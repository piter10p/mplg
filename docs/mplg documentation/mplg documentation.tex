\documentclass[a4paper]{report}
\title{Math Programming Language Documentation}
\author{Piotr Paszko}
\date{2018}

\begin{document}

\maketitle

\begin{abstract}
One day I've tried to teach person close to me basics of programming. After few hours I've realised, that the best method of explanation is by presenting code in mathematical syntax. Then I thought there's no easy to learn programming language, that uses syntax similar to mathematical and easy to read.
\end{abstract}

\tableofcontents

\chapter{Introduction}
\section{Language Assumption}
MPLG have to be easy to learn for people without IT education, but with basic mathematical knowledge. The language have to provide clear and easy environment for both basic mathematical algorithms and complex engineering calculations. It have to be distributed with IDE adapted for it (providing support for all functions). It have to be executed by interpreter.

\section{Language Goals}
MPLG have to be able to meet the following goals:
\begin{enumerate}
  \item The language have to be easy to learn for people without IT education, but with mathematical education.
  \item The language have to be easy to read, with clear syntax.
  \item The language does not have to be able to handle both mathematical algorithms and complex engineering calculations. 
  \item The language have to be distributed with IDE adapted for it.
  \item The language have to provide simple library system.
  \item The language have to be designed to be executed by interpreter.
\end{enumerate}

\section{What the language was not designed for}
MPLG was not designed to:
\begin{enumerate}
  \item Creating software (the language does not have to provide API interacting with system, computer components etc.).
  \item Creating big and complex systems (language syntax was designed only for calculations, it's not providing testing systems, OOP etc.).
\end{enumerate}
\end{document}
